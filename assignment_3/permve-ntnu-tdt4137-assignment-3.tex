%%%%%%%%%%%%%%%%%%%%%%%%%%%%%%%%%%%%%%%%%
% Short Sectioned Assignment
% LaTeX Template
% Version 1.0 (5/5/12)
%
% This template has been downloaded from:
% http://www.LaTeXTemplates.com
%
% Original author:
% Frits Wenneker (http://www.howtotex.com)
%
% License:
% CC BY-NC-SA 3.0 (http://creativecommons.org/licenses/by-nc-sa/3.0/)
%
%%%%%%%%%%%%%%%%%%%%%%%%%%%%%%%%%%%%%%%%%

%----------------------------------------------------------------------------------------
%	PACKAGES AND OTHER DOCUMENT CONFIGURATIONS
%----------------------------------------------------------------------------------------

\documentclass[paper=a4, fontsize=11pt]{scrartcl} % A4 paper and 11pt font size

\usepackage[usenames,dvipsnames,x11names]{xcolor} % Colors

\usepackage[T1]{fontenc} % Use 8-bit encoding that has 256 glyphs
\usepackage{fourier} % Use the Adobe Utopia font for the document - comment this line to return to the LaTeX default
\usepackage[english]{babel} % English language/hyphenation
\usepackage{amsmath,amsfonts,amsthm} % Math packages

\usepackage{sectsty} % Allows customizing section commands
\allsectionsfont{\centering \normalfont\scshape} % Make all sections centered, the default font and small caps

\usepackage{tikz}
\usepackage{pgfplots}
\usetikzlibrary{plotmarks}

\usepackage{tabularx}  % for 'tabularx' environment and 'X' column type
\usepackage{ragged2e}  % for '\RaggedRight' macro (allows hyphenation)
\newcolumntype{Y}{>{\RaggedRight\arraybackslash}X} 
\usepackage{booktabs}  % for \toprule, \midrule, and \bottomrule macros 

\usepackage{enumitem}
\usepackage{lastpage}
\usepackage{multirow}
\usepackage{acronym}
\usepackage{graphicx} % Allow use of images
\usepackage{fancyhdr} % Custom headers and footers
\pagestyle{fancyplain} % Makes all pages in the document conform to the custom headers and footers
\fancyhead{} % No page header - if you want one, create it in the same way as the footers below
\fancyfoot[L]{} % Empty left footer
\fancyfoot[C]{} % Empty center footer
\fancyfoot[C]{\thepage~of~\pageref{LastPage}} % Page numbering for right footer
\renewcommand{\headrulewidth}{0pt} % Remove header underlines
\renewcommand{\footrulewidth}{0pt} % Remove footer underlines
\setlength{\headheight}{13.6pt} % Customize the height of the header

\setlength\parindent{0pt} % Removes all indentation from paragraphs - comment this line for an assignment with lots of text

%----------------------------------------------------------------------------------------
%	TITLE SECTION
%----------------------------------------------------------------------------------------

\newcommand{\horrule}[1]{\rule{\linewidth}{#1}} % Create horizontal rule command with 1 argument of height



\title{	
\normalfont \normalsize 
\textsc{Norwegian University of Science and Technology\\TDT4137 -- Cognitive Architectures}
\horrule{0.5pt} \\[0.4cm]
\huge Assignment 3:\\ Artificial Neural Networks\\
\horrule{2pt} \\[0.5cm]
}

\author{Per Magnus Veierland\\permve@stud.ntnu.no}

\date{\normalsize\today}

\begin{document}

\maketitle

\begin{enumerate}
\item The \textsc{MATLAB}-function \texttt{perceptron\_train} constructs a set of weights for a single perceptron when given a set of training inputs and outputs; as well as a threshold value~($\theta$) and a learning rate~($\alpha$). It will continue updating the weights according to the perceptron training rule until either a full epoch has been completed without any errors or until the maximum number of epochs has been reached.

\item Table~\ref{table:and} and Table~\ref{table:or} shows the intermediary and final outputs when applying the \texttt{perceptron\_train} function on the truth tables of an \textsc{AND}-function and an \textsc{OR}-function.

Changing the initial random weights will result in slightly different final weights.

For both the \textsc{AND}-input and the \textsc{OR}-input; increasing the threshold value~($\theta$) will keep increasing both of the final weights and the training function will keep finding a solution while requiring more epochs to complete. Decreasing the threshold value to 0 or below will prevent the training function from finding a solution.

\item This problem was solved using the \textsc{MATLAB} Neural Network Toolbox. The \textit{Levenberg-Marquardt} backpropagation algorithm was used to train the network.

Since the \textsc{MATLAB} implementation uses both a training set, a validation set and a test set -- eight elements is a very sparse dataset. To make the implementation work better with the dataset it was simply repeated such that the number of input elements is 40 instead of 8 with redundant input data repeated 5 times.

Reducing the number of hidden nodes significantly increased the time it took to train the network. However even when reducing the number of hidden nodes to 1; the resulting network was able to perform well and produce the expected outputs for the entire input range with no error greater than 0.001.

The training has established the following values for the network:

\begin{tabular}{ll}
Input weight matrix & \texttt{[[-0.0274];[]]} \\
Layer weight matrix & \texttt{[[] []; [-36.4661]]} \\
Bias vectors        & \texttt{[[8.3433e-06];[3.0332e-04]]} \\
\end{tabular}

The network with 1 hidden node works well for numbers outside the training domain. For inputs down to -25 and up to 33 the output has an error less than 0.5. With further adjustments to the training the network can likely be made to accommodate a much greater range.

\end{enumerate}

\begin{table}
\centering
\begin{tabular}{cccccccccc}
\toprule
& & & Desired  & \multicolumn{2}{c}{Initial} & Actual & & \multicolumn{2}{c}{Final} \\
& \multicolumn{2}{c}{Inputs} & output & \multicolumn{2}{c}{weights} & output & Error & \multicolumn{2}{c}{weights} \\
Epoch & $x_1$ & $x_2$ & $Y_d$ & $w_1$ & $w_2$ & $Y$ & $e$ & $w_1$ & $w_2$ \\
\midrule
1 & 0 & 0 & 0 & 0.191 & 0.428 & 0 & 0 & 0.191 & 0.428 \\
  & 0 & 1 & 0 & 0.191 & 0.428 & 1 & -1 & 0.191 & 0.328 \\
  & 1 & 0 & 0 & 0.191 & 0.328 & 0 & 0 & 0.191 & 0.328 \\
  & 1 & 1 & 1 & 0.191 & 0.328 & 1 & 0 & 0.191 & 0.328 \\
2 & 0 & 0 & 0 & 0.191 & 0.328 & 0 & 0 & 0.191 & 0.328 \\
  & 0 & 1 & 0 & 0.191 & 0.328 & 1 & -1 & 0.191 & 0.228 \\
  & 1 & 0 & 0 & 0.191 & 0.228 & 0 & 0 & 0.191 & 0.228 \\
  & 1 & 1 & 1 & 0.191 & 0.228 & 1 & 0 & 0.191 & 0.228 \\
3 & 0 & 0 & 0 & 0.191 & 0.228 & 0 & 0 & 0.191 & 0.228 \\
  & 0 & 1 & 0 & 0.191 & 0.228 & 1 & -1 & 0.191 & 0.128 \\
  & 1 & 0 & 0 & 0.191 & 0.128 & 0 & 0 & 0.191 & 0.128 \\
  & 1 & 1 & 1 & 0.191 & 0.128 & 1 & 0 & 0.191 & 0.128 \\
4 & 0 & 0 & 0 & 0.191 & 0.128 & 0 & 0 & 0.191 & 0.128 \\
  & 0 & 1 & 0 & 0.191 & 0.128 & 0 & 0 & 0.191 & 0.128 \\
  & 1 & 0 & 0 & 0.191 & 0.128 & 0 & 0 & 0.191 & 0.128 \\
  & 1 & 1 & 1 & 0.191 & 0.128 & 1 & 0 & 0.191 & 0.128 \\
\bottomrule
\end{tabular}
\caption{Perceptron learning results for \textsc{AND}-input.}
\label{table:and}
\end{table}

\begin{table}
\centering
\begin{tabular}{cccccccccc}
\toprule
& & & Desired  & \multicolumn{2}{c}{Initial} & Actual & & \multicolumn{2}{c}{Final} \\
& \multicolumn{2}{c}{Inputs} & output & \multicolumn{2}{c}{weights} & output & Error & \multicolumn{2}{c}{weights} \\
Epoch & $x_1$ & $x_2$ & $Y_d$ & $w_1$ & $w_2$ & $Y$ & $e$ & $w_1$ & $w_2$ \\
\midrule
1 & 0 & 0 & 0 & 0.060 & 0.682 & 0 & 0 & 0.060 & 0.682 \\
  & 0 & 1 & 1 & 0.060 & 0.682 & 1 & 0 & 0.060 & 0.682 \\
  & 1 & 0 & 1 & 0.060 & 0.682 & 0 & 1 & 0.160 & 0.682 \\
  & 1 & 1 & 1 & 0.160 & 0.682 & 1 & 0 & 0.160 & 0.682 \\
2 & 0 & 0 & 0 & 0.160 & 0.682 & 0 & 0 & 0.160 & 0.682 \\
  & 0 & 1 & 1 & 0.160 & 0.682 & 1 & 0 & 0.160 & 0.682 \\
  & 1 & 0 & 1 & 0.160 & 0.682 & 0 & 1 & 0.260 & 0.682 \\
  & 1 & 1 & 1 & 0.260 & 0.682 & 1 & 0 & 0.260 & 0.682 \\
3 & 0 & 0 & 0 & 0.260 & 0.682 & 0 & 0 & 0.260 & 0.682 \\
  & 0 & 1 & 1 & 0.260 & 0.682 & 1 & 0 & 0.260 & 0.682 \\
  & 1 & 0 & 1 & 0.260 & 0.682 & 1 & 0 & 0.260 & 0.682 \\
  & 1 & 1 & 1 & 0.260 & 0.682 & 1 & 0 & 0.260 & 0.682 \\
\bottomrule
\end{tabular}
\caption{Perceptron learning results for \textsc{OR}-input.}
\label{table:or}
\end{table}

\end{document}

