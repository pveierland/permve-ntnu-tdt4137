\input{permve-ntnu-latex-assignment.tex}

\title{
\normalfont \normalsize
\textsc{Norwegian University of Science and Technology\\TDT4137 -- Cognitive Architectures}
\horrule{0.5pt} \\[0.4cm]
\huge Assignment 1:\\ MHP / Fitt's Law\\
\horrule{2pt} \\[0.5cm]
}

\author{Per Magnus Veierland\\permve@stud.ntnu.no}

\date{\normalsize\today}

\newacro{ID}{Index of Difficulty}
\newacro{LTM}{Long-Term Memory}
\newacro{MHP}{Model Human Processor}
\newacro{MT}{Movement Time}
\newacro{VIS}{Visual Image Store}
\newacro{VR}{Virtual Reality}

\begin{document}

\maketitle

\begin{enumerate}[label=\alph*)]

\item \textbf{A brake pedal shall be pushed down as soon as the red braking light from the car in front lights up. Calculate using \ac{MHP} what the response time will be until the braking pedal is pushed down.}

The problem described can be analyzed as a simple reaction scenario using the \ac{MHP}. It is assumed that the person controlling the car is focused on the traffic ahead of him at the time when the braking light in front is activated. At time zero, the braking light on the car in front is activated, denoted as the physical signal $\alpha$. The driver's \textit{Perceptual Processor} will register the activated braking light and construct a physically coded representation, $\alpha'$, in the \ac{VIS}. Shortly after being constructed in the \ac{VIS}, a visually coded representation, $\alpha''$, will appear in the \textit{Working Memory}. The total time spent up until this point is one \textit{Perceptual Processor} cycle, denoted by $\tau_p$.

As part of the \textit{Recognize-Act} cycle, the \textit{Cognitive Processor} will initiate an action in \textit{Long-Term Memory} which is associated with the visual brake light symbol ($\alpha''$) and produce the motor response \textsc{PUSH-BRAKE} which is placed in the \textit{Working Memory}. This operation takes a single \textit{Cognitive Processor} cycle, denoted by $\tau_C$.

After the \textsc{PUSH-BRAKE} motor command is placed in \textit{Working Memory}, the \textit{Motor Processor} will act on the available input and execute the \textsc{PUSH-BRAKE} action. This takes a single \textit{Motor Processor} cycle, denoted by $\tau_M$.

\clearpage

The total response time from the red brake light of the car in front lights up until the driver is able to step on the brake pedal is (Table~\ref{tab:a}):
\begin{align*}
\textit{ResponseTime}
&= \tau_p + \tau_C + \tau_M \\
&= 100~[50 \sim 200]~\text{msec} + 70~[25 \sim 170]~\text{msec} + 70~[30 \sim 100]~\text{msec} \\
&= 240~[105 \sim 470]~\text{msec}
\end{align*}

\end{enumerate}

\begin{table}
{\footnotesize\centering
\begin{tabular}{cllllll}
\toprule
Step & Description                    & Environment & VIS       & WM                              & Body                & Elapsed Time               \\
\midrule
1.   & Brake light appears            & $\alpha$    &           &                                 &                     & 0                          \\
2.   & Transmitted to \ac{VIS} and WM & $\alpha$    & $\alpha'$ & $\alpha''$                      &                     & $\tau_p$                   \\
3.   & Initiate brake motor response  & $\alpha$    & $\alpha'$ & $\alpha''$. \textsc{PUSH-BRAKE} &                     & $\tau_p + \tau_C$          \\
4.   & Process brake motor command    & $\alpha$    & $\alpha'$ & $\alpha''$. \textsc{PUSH-BRAKE} & \textsc{PUSH-BRAKE} & $\tau_p + \tau_C + \tau_M$ \\
\bottomrule
\end{tabular}
}
\caption{Simple reaction response}
\label{tab:a}
\end{table}

\begin{enumerate}[resume*]
\item \textbf{Assume that a user views an image of a flag on a screen. How long will it take until she recognizes that it is Scandinavian? (Assume that the flag's semantic name must be retrieved from \ac{LTM}.)}

It is here assumed that \textsc{DanishFlag} has an abstract code stored in the user's long-term memory and that \textsc{Scandinavian} is a known object class.

Following the cognitive processing rates set by Cavanaugh (1972) a flag is likely a more complicated symbol than a number, but is still composed of basic geometric shapes and clear colors. The time it takes to recognize a flag is therefore likely longer than a plain geometric shape (50~msec), but shorter than a random form (68~msec). In the following calculation the recognition and classification are both assumed to consume one \textit{Cognitive Processor} cycle each.

The total time it takes a user to recognize a flag as Scandinavian is (Table~\ref{tab:b}):
\begin{align*}
\textit{ClassificationTime}
&= \tau_p + 2 \cdot \tau_C \\
&= 100~[50 \sim 200]~\text{msec} + 2 \cdot 70~[25 \sim 170]~\text{msec} \\
&= 240~[100 \sim 540]~\text{msec}
\end{align*}

\end{enumerate}

\begin{table}
{\footnotesize\centering
\begin{tabular}{cllllll}
\toprule
Step & Description                    & Environment & VIS       & WM                                                       & Elapsed Time              \\
\midrule
1.   & Danish flag appears            & $\alpha$    &           &                                                          & 0                         \\
2.   & Transmitted to \ac{VIS} and WM & $\alpha$    & $\alpha'$ & $\alpha''$                                               & $\tau_p$                  \\
3.   & Recognize Danish flag          & $\alpha$    & $\alpha'$ & $\alpha''$ : \textsc{DanishFlag}                         & $\tau_p + \tau_C$         \\
4.   & Classify as Scandinavian       & $\alpha$    & $\alpha'$ & $\alpha''$ : \textsc{DanishFlag} : \textsc{Scandinavian} & $\tau_p + 2 \cdot \tau_C$ \\
\bottomrule
\end{tabular}
}
\caption{Flag classification}
\label{tab:b}
\end{table}

\begin{enumerate}[resume*]
\item \textbf{What is meant by \acf{ID}? Use Fitt's law to show that it is faster to move the cursor to a target which is on the edge of the screen compared to a target which is positioned closed to the edge of the screen}

\acf{ID} is a metric used to quantify the difficulty of making a controlled movement a certain distance $D$ to a target of width $W$. ``Fitt's Law'' is named after Paul Fitts who proposed in his 1954 paper that this difficulty was proportional to twice the distance moved and inversely proportional to the width of the target, formulated as:

\begin{displaymath}
\text{ID} = \log_2\big(\frac{2D}{W}\big)
\end{displaymath}

The most commonly used formulation today is the Shannon formulation proposed by Scott MacKenzie:

\begin{displaymath}
\text{ID} = \log_2\big(\frac{D}{W} + 1\big)
\end{displaymath}

The \ac{ID} is commonly used in a linear regression model to calculate the \ac{MT}:

\begin{displaymath}
\text{MT} = a + b \cdot \text{ID}
\end{displaymath}

From the question description we are using the model values $a = 50$ and $b = 150$.

Using Windows, there is a 5~mm tall menu ($W = 5$~mm) which is placed a short distance away from the edge of the screen. The average distance of movement to reach this menu is $D = 80$~mm. The average moving time to reach the menu from a point on the screen is:

\begin{displaymath}
\text{MT}_\text{Windows} = a + b \cdot \log_2\Big(\frac{D}{W} + 1\Big) = 50~{\footnotesize \frac{\text{ms}}{\text{bit}}} + 150~{\footnotesize \frac{\text{ms}}{\text{bit}}} \cdot \log_2\Big(\frac{80~\text{\footnotesize mm}}{5~\text{\footnotesize mm}} + 1\Big) \approx 663~\text{\footnotesize ms}
\end{displaymath}

Using Macintosh, the menu is also 5~mm tall. However this menu is placed on the edge of the screen which means we can use a value $W = 50$~mm. The average distance of movement is the same. The average moving time to reach the menu from a point on the screen is:

\begin{displaymath}
\text{MT}_\text{Macintosh} = a + b \cdot \log_2\Big(\frac{D}{W} + 1\Big) = 50~{\footnotesize \frac{\text{ms}}{\text{bit}}} + 150~{\footnotesize \frac{\text{ms}}{\text{bit}}} \cdot \log_2\Big(\frac{80~\text{\footnotesize mm}}{50~\text{\footnotesize mm}} + 1\Big) \approx 257~\text{\footnotesize ms}
\end{displaymath}

The evaluation shows that a menu placed along the border of a screen can be reached in less than half the time compared to a menu placed a short distance away from the border.

\clearpage

\item \textbf{How many images per second must be shown to create an illusion of continuity in time?}

Within the \ac{MHP} framework, perceptual events which occur within a single \textit{Perceptual Processor} cycle will be combined to a single percept. This phenomena is found with both visual and acoustic percepts and requires that the percepts are closely related.

To create an illusion of continuity for images each pair of frames must appear within a single \textit{Perceptual Processor} cycle, denoted by $\tau_p$. Based on this, we can calculate the minimum image rate required:
\begin{displaymath}
\textit{ImageRate} > \Bigg[ \frac{1}{\tau_p} = \frac{1}{100~[50 \sim 200]~\text{msec}} = 10~[20 \sim 5]~\text{Hz} \Bigg]
\end{displaymath}

For the illusion to work for all users, the image rate must exceed 20~Hz.

This result will be affected by the \textit{Variable Perceptual Processor Rate Principle} (P1) which states that the \textit{Perceptual Processor} cycle time, $\tau_p$, varies inversely with stimulus intensity. This means that for intense stimulus, e.g. immersive gaming using \ac{VR} glasses, the image rate will need to be higher to guarantee the illusion of continuity.
\end{enumerate}

\end{document}
