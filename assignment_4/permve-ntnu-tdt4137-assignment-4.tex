\input{../templates/assignment.tex}

\usepackage{mathtools}
\usepackage{relsize}

% Default fixed font does not support bold face
\DeclareFixedFont{\ttb}{T1}{txtt}{bx}{n}{10} % for bold
\DeclareFixedFont{\ttm}{T1}{txtt}{m}{n}{10}  % for normal

% Custom colors
\definecolor{deepblue}{rgb}{0,0,0.5}
\definecolor{deepred}{rgb}{0.6,0,0}
\definecolor{deepgreen}{rgb}{0,0.5,0}
\definecolor{lbcolor}{rgb}{0.99,0.99,0.99}

\usepackage{listings}

% Python style for highlighting
\newcommand\pythonstyle{\lstset{
language=Python,
basicstyle=\ttm,
numbers=left,
otherkeywords={self},
keywordstyle=\ttb\color{deepblue},
stringstyle=\color{deepgreen},
backgroundcolor=\color{lbcolor},
frame=single,
showstringspaces=false
}}

% Python environment
\lstnewenvironment{python}[1][]
{
\pythonstyle
\lstset{#1}
}
{}

% Python for external files
\newcommand\pythonexternal[2][]{{
\pythonstyle
\lstinputlisting[#1]{#2}}}

% Python for inline
\newcommand\pythoninline[1]{{\pythonstyle\lstinline!#1!}}

\title{
\normalfont \normalsize
\textsc{Norwegian University of Science and Technology\\TDT4137 -- Cognitive Architectures}
\horrule{0.5pt} \\[0.4cm]
\huge Assignment 4:\\ Fuzzy Reasoning\\
\horrule{2pt} \\[0.5cm]
}

\author{Per Magnus Veierland\\permve@stud.ntnu.no}

\date{\normalsize\today}

\newacro{COG}{Centre of Gravity}

\begin{document}

\maketitle

\section*{Fuzzy sets and rules}

\begin{enumerate}[label=\alph*)]
\item

The distance between the robot car and the car in front is 3.6. This value intersects with the linguistic sets \textsc{Small} and \textsc{Perfect}:

\begin{displaymath}
\mu_{\textsc{Small}} = \frac{4.5 - 3.6}{4.5 - 3.0} = 0.6
\end{displaymath}

\begin{displaymath}
\mu_{\textsc{Perfect}} = \frac{3.6 - 3.5}{5.0 - 3.5} \approx 0.07
\end{displaymath}

The gradient of the distance is the delta variable which has the value 1.1 and belongs to the linguistic sets \textsc{Stable} and \textsc{Growing}:

\begin{displaymath}
\mu_{\textsc{Stable}} = \frac{1.5 - 1.1}{1.5 - 0} \approx 0.27
\end{displaymath}

\begin{displaymath}
\mu_{\textsc{Growing}} = \frac{1.1 - 0.5}{2 - 0.5} = 0.4
\end{displaymath}

\begin{enumerate}[label=\textsc{Rule \arabic*}:]
\item \textbf{If} \textsc{Distance} \textbf{is} \textsc{Small} \textbf{and} \textsc{Delta} \textbf{is} \textsc{Growing} \textbf{then} \textsc{Action} \textbf{is} \textsc{None}

\begin{displaymath}
\mu_{\textsc{None}} = \min(\mu_{\textsc{Small}}, \mu_{\textsc{Growing}}) = \min(0.6, 0.4) = 0.4
\end{displaymath}

\item \textbf{If} \textsc{Distance} \textbf{is} \textsc{Small} \textbf{and} \textsc{Delta} \textbf{is} \textsc{Stable} \textbf{then} \textsc{Action} \textbf{is} \textsc{SlowDown}

\begin{displaymath}
\mu_{\textsc{SlowDown}} = \min(\mu_{\textsc{Small}}, \mu_{\textsc{Stable}}) = \min(0.6, 0.27) = 0.27
\end{displaymath}

\item \textbf{If} \textsc{Distance} \textbf{is} \textsc{Perfect} \textbf{and} \textsc{Delta} \textbf{is} \textsc{Growing} \textbf{then} \textsc{Action} \textbf{is} \textsc{SpeedUp}

\begin{displaymath}
\mu_{\textsc{SpeedUp}} = \min(\mu_{\textsc{Perfect}}, \mu_{\textsc{Growing}}) = \min(0.07, 0.4) = 0.07
\end{displaymath}

\item \textbf{If} \textsc{Distance} \textbf{is} \textsc{VeryBig} \textbf{and} \big(\textsc{Delta} \textbf{is not} \textsc{Growing} \textbf{or} \textsc{Delta} \textbf{is not} \textsc{GrowingFast}\big) \textbf{then} \textsc{Action} \textbf{is} \textsc{FloorIt}

\begin{align*}
\mu_{\textsc{FloorIt}} &= \min\big(\mu_{\textsc{VeryBig}}, \max(\neg\mu_{\textsc{Growing}}, \neg\mu_{\textsc{GrowingFast}})\big)\\
&= \min\big(0, \max(1 - 0.4, 1 - 0)\big)\\
&= \min\big(0, \max(0.6, 1)\big)\\
&= \min(0, 1)\\
&= 0
\end{align*}

\item \textbf{If} \textsc{Distance} \textbf{is} \textsc{VerySmall} \textbf{then} \textsc{Action} \textbf{is} \textsc{BrakeHard}

\begin{displaymath}
\mu_{\textsc{BrakeHard}} = \mu_{\textsc{VerySmall}} = 0
\end{displaymath}

\end{enumerate}

The resulting defuzzified output can be calculated as the \ac{COG} of the aggregated fuzzy set $A$. Rule outputs are clipped before they are aggregated.

%Raw data:
%-10 = 0
%-9  = 0
%-8  = 0
%-7  = 0
%-6  = 0.2
%-5  = 0.2
%-4  = 0.2
%-3  = 0.2
%-2  = 0.3
%-1  = 0.4
%0   = 0.4
%1   = 0.4
%2   = 0.3
%3   = 0.07
%4   = 0.07
%5   = 0.07
%6   = 0.07
%7   = 0
%8   = 0
%9   = 0
%10  = 0

\begin{align*}
\text{COG} &= \frac{\mathlarger{\int_{-10}^{10} x \cdot \mu_A(x)~dx}}{\mathlarger{\int_{-10}^{10} \mu_A(x)~dx}}\\[0.2cm]
&\approx \frac{\sum\limits_{n=-10}^{-7}(n) \cdot 0 + \sum\limits_{n=-6}^{-3}(n) \cdot 0.2 + (-2) \cdot 0.3 + \sum\limits_{n=-1}^{1}(n) \cdot 0.4 + (2) \cdot 0.3 + \sum\limits_{n=3}^{6}(n) \cdot 0.07 + \sum\limits_{n=7}^{10}(n) \cdot 0}{4 \cdot 0 + 4 \cdot 0.2 + 0.3 + 3 \cdot 0.4 + 0.3 + 4 \cdot 0.07 + 4 \cdot 0}\\[0.2cm]
&= \frac{0 + -3.6 + -0.6 + 0 + 0.6 + 1.26 + 0}{2.88} = \frac{-2.34}{2.88} = -0.8125
\end{align*}

\item
Performing \textit{Mamdani} reasoning with \ac{COG} defuzzification using segmentation steps yielded an output of -1.1066.

\newpage

\pythonexternal{fuzzy.py}

\end{enumerate}

\end{document}

