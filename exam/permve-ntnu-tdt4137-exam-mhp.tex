\cardfrontfoot{Cognitivism}

\begin{flashcard}[Question]{What is the \textbf{Physical Symbol System Hypothesis}?}
\begin{center}
\textit{``A physical symbol system has the necessary and sufficient means for general intelligent action.''}

Newell and Simon
\end{center}
\end{flashcard}

\begin{flashcard}[Question]{What is the \textbf{Heuristic Search Hypothesis}?}
\begin{center}
\textit{``The solutions to problems are represented as symbol structures. A physical-symbol system exercises its intelligence in problem-solving by search, that is, by generating and progressively modifying symbol structures until it produces a solution structure.''}

Newell and Simon
\end{center}
\end{flashcard}

\begin{flashcard}[Question]{What is the \textbf{Model Human Processor} and how is it described?}
\begin{center}
The \textbf{Model Human Processor} is a system-level model of the human mind which views it as an information-processing system. It was designed to help predict human behavior in the context of human-computer interaction when designing interfaces.

\medskip

It is described by a set of memories and processors, together with a set of principles, also known as ``the princriples of operation''.
\end{center}
\end{flashcard}

\begin{flashcard}[Question]{What are the three subsystems of the \textbf{Model Human Processor} and what are their functions?}
\begin{center}
\begin{itemize}
\item The \textit{perceptual system}, which consists of visual and auditory sensors together with their \textit{Visual Image Store} and \textit{Auditory Image Store} buffers which holds the output of the sensory system while it is being symbolically decoded.
\item The \textit{cognitive system}, which receives symbolically decoded infromation from the sensory image stores in its \textit{Working Memory} and uses previously stored information in \textit{Long-Term Memory} to generate a response which is stored back into the \textit{Working Memory}.
\item The \textit{motor system}, which receives motor commands from the \textit{Working Memory} and carries out the motor action.
\end{itemize}
\end{center}
\end{flashcard}

\begin{flashcard}[Question]{What are the parameters of memories and processors in the \textbf{Model Human Processor}?}
\begin{center}
Memory parameters:
\begin{itemize}
\item $\mu$, the storage capacity in items,
\item $\delta$, the decay time of an item, and
\item $\kappa$, the main code type; physical, acoustic, visual, or semantic.
\end{itemize}
Processor parameters:
\begin{itemize}
\item $\tau$, the cycle time.
\end{itemize}
\end{center}
\end{flashcard}

\begin{flashcard}[Question]{What are the 10 principes of operation of the \textbf{Model Human Processor}?}
{\small
\begin{enumerate}[start=0,label=\textbf{P\arabic*.}]
\item Recognize-Act Cycle of the Cognitive Processor
\item Variable Perceptual Processor Rate Principle
\item Encoding Specificity Principle
\item Discrimination Principle
\item Variable Cognitive Processor Rate Principle
\item Fitt's Law
\item Power Law of Practice
\item Uncertainty Principle
\item Rationality Principle
\item Problem Space Principle
\end{enumerate}
}
\end{flashcard}

\begin{flashcard}[Question]{What is the \textbf{Recognize-Act Cycle of the Cognitive Processor}?}
\begin{center}
On each cycle of the \textit{Cognitive Processor}, the contents of the \textit{Working Memory} initiate actions associatively linked to them in \textit{Long-Term Memory}; these actions in turn modify the contents of \textit{Working Memory}.
\end{center}
\end{flashcard}

\begin{flashcard}[Question]{What is the \textbf{Variable Perceptual Processor Rate Principle}?}
\begin{center}
The \textit{Perceptual Processor} cycle time $\tau_p$ varies inversely with stimulus intensity.
\end{center}
\end{flashcard}

\begin{flashcard}[Question]{What is the \textbf{Encoding Specificity Principle}?}
\begin{center}
Specific encoding operations performed on what is perceived determine what is stored, and what is stored determines what retrieval cues are effective in providing access to what is stored.
\end{center}
\end{flashcard}

\begin{flashcard}[Question]{What is the \textbf{Discrimination Principle}?}
\begin{center}
The difficulty of memory retrieval is determined by the candidates that exist in the memory, relative to the retrieval clues.
\end{center}
\end{flashcard}

\begin{flashcard}[Question]{What is the \textbf{Variable Cognitive Processor Rate Principle}?}
\begin{center}
The \textit{Cognitive Processor} cycle time $\tau_c$ is shorter when greater effort is induced by increased task demands or information loads; it also diminishes with practice.
\end{center}
\end{flashcard}

\begin{flashcard}[Question]{What is \textbf{Fitt's Law}?}
\begin{center}
The time $T_{\text{pos}}$ to move the hand to a target of size $S$\\which lies a distance $D$ away is given by:
\begin{displaymath}
T_{\text{pos}} = I_M \log_2 (\frac{D}{S} + 0.5),
\end{displaymath}
where $I_M = 100~[70 \sim 120] \frac{\text{msec}}{\text{bit}}$.
\end{center}
\end{flashcard}

\begin{flashcard}[Question]{What is the \textbf{Power Law of Practice}?}
\begin{center}
The time $T_n$ to perform a task on the $n$th trial follows a power law:
\begin{displaymath}
T_n = \frac{T_1}{n^\alpha},
\end{displaymath}
where $\alpha = 0.4~[0.2 \sim 0.6]$.
\end{center}
\end{flashcard}

\begin{flashcard}[Question]{What is the \textbf{Uncertainty Principle}?}
\begin{center}
The decision time $T$ increases with uncertainty about the\\judgement or decision to be made:
\begin{displaymath}
T = I_C H,
\end{displaymath}
where $H$ is the information-theoretic entropy of the decision and $I_C = 150~[0 \sim 157] \frac{\text{msec}}{\text{bit}}$. For $n$ equally probable alternatives (\textit{Hick's Law});
\begin{displaymath}
H = \log_2(n + 1).
\end{displaymath}
For $n$ alternatives with different probabilities, $p_i$, of occurrence;
\begin{displaymath}
H = \sum_i p_i \log_2 (\frac{1}{p_i} + 1).
\end{displaymath}
\end{center}
\end{flashcard}

\begin{flashcard}[Question]{What is the \textbf{Rationality Principle}?}
\begin{center}
A person acts so as to attain his goals through rational action, given the structure of the task and his inputs of information and bounded by limitations on his knowledge and processing ability:
\begin{align*}
\text{Goals} + \text{Task} + \text{Operators} + \text{Inputs}\\ + \text{Knowledge} + \text{Process-limits} \rightarrow \text{Behavior}
\end{align*}
\end{center}
\end{flashcard}

\begin{flashcard}[Question]{What is the \textbf{Problem Space Principle}?}
\begin{center}
The rational activity in which people engage to solve a problem can be described in terms of
\begin{enumerate*}
\item a set of states of knowledge,
\item operators for changing one state into another,
\item constraints on applying operators, and
\item control knowledge for deciding which operator to apply next.
\end{enumerate*}
\end{center}
\end{flashcard}

\begin{flashcard}[Question]{How is decay time for memory defined?}
\begin{center}
Decay time is defined as the time after which the probability of retrieval is less than 50\%.
\end{center}
\end{flashcard}

\begin{flashcard}[Question]{At what time is a sensor stimulus available in the \textbf{Visual Image Store} or \textbf{Auditory Image Store}?}
\begin{center}
If a stimulus occurs at $t = 0$, then at the end of time $t = t_p$ the image will be available in the buffer store.
\end{center}
\end{flashcard}

\begin{flashcard}[Question]{What is \textbf{Bloch's Law}?}
\begin{center}
\textbf{Bloch's Law} states that a brief pulse of light of duration $t$ and intensity $I$, has the same appearance as a longer pulse of less intense light, provided both pulses lasts less than the perceptual processor's cycle time, $t_p$:

\begin{displaymath}
I \cdot t = k, t < \tau_p
\end{displaymath}

\medskip

Sufficiently similar perceptual events within a single perceptual cycle are combined into a single percept.
\end{center}
\end{flashcard}

\begin{flashcard}[Question]{What is the relationship between \textbf{Working Memory} and \textbf{Long-Term Memory}, and what are their main code types?} 
\begin{center}
\textbf{Working Memory} consists of a subset of the elements in \textbf{Long-Term Memory} which have become activated.

\medskip

The main code types of \textbf{Working Memory} are \textit{acoustic} and \textit{visual}, while for \textbf{Long-Term Memory} the main code type is \textit{semantic}.
\end{center}
\end{flashcard}

\begin{flashcard}[Question]{What is a \textit{chunk}?}
\begin{center}
A \textit{chunk} is a symbol in \textbf{Working Memory} representing a concept such as a letter or a word. \textit{Chunks} can themselves be composites of other \textit{chunks}, e.g. the letters ``I'', ``B'', ``M'' are each \textit{chunks} which can be represented using a single \textit{chunk}; ``IBM''.

\medskip

\textit{Chunks} can be related to other \textit{chunks}. The \textit{chunk} \textsc{Robin} is a subset of the \textit{chunk} \textsc{Bird}, it has \textit{chunks} \textsc{Wings}, and it can \textit{chunk} \textsc{Fly}. The activation of a \textit{chunk} spreads to related chunks.
\end{center}
\end{flashcard}

\begin{flashcard}[Question]{What is the difference between the \textit{pure capacity} and the \textit{effective capacity} of \textbf{Working Memory}?}
\begin{center}
The \textit{pure capacity} of \textbf{Working Memory} can be found by testing the number of immidiate preceding digits recallable from a long series when the series unexpectedly stops.
\begin{displaymath}
\mu_\text{WM} = 3~[2.5 \sim 4.1]~\text{chunks}
\end{displaymath}
The \textit{effective capacity} of \textbf{Working Memory} is its capacity when augmented by the use of \textbf{Long-Term Memory}, e.g. the longest number that can be repeated back.
\begin{displaymath}
\mu_\text{WM}^* = 7~[5 \sim 9]~\text{chunks}
\end{displaymath}
\end{center}
\end{flashcard}

\begin{flashcard}[Question]{How are added into \textbf{Long-Term Memory}?}
\begin{center}
Items cannot be added directly to \textbf{Long-Term Memory}, rather items in \textbf{Working Memory} have a certain probability of being retrievable later from \textbf{Long-Term Memory}. The more associations the item has, the greater the probability of being retrieved. If a user wants to remember something later, his best strategy is to attemt to associate it with items already in \textbf{Long-Term Memory}, especially in novel ways so there is unlikely to be interference with other items.

\medskip

The probability that an item will be stored in \textbf{Long-Term Memory} increases with residence time in \textbf{Working Memory}, up until the decay time of the \textbf{Working Memory}. The \textbf{Long-Term Memory} acts as a fast-read, slow-write system.
\end{center}
\end{flashcard}

\begin{flashcard}[Question]{What is the \textbf{recognize-act cycle}?}
\begin{center}
At each cycle, the \textbf{Cognitive Processor} matches the contents of the \textbf{Working Memory} with associated actions in \textbf{Long-Term Memory}, placing the results of an action back into \textbf{Working Memory}.
\end{center}
\end{flashcard}
