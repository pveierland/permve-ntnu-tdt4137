\cardfrontfoot{Intro}

\begin{flashcard}[Question]{What are the four aspects of modelling cognitive systems?}
\begin{itemize}
\item The computational / bio-inspired spectrum.
\item The level of abstraction in the biological model.
\item The mutual dependence of brain, body, and environment.
\item The ultimate-proximate distinction.
\end{itemize}
\end{flashcard}

\begin{flashcard}[Question]{What is the Ultimate-Proximate Distinction?}
\begin{center}
Several behaviors can be realized with the same mechanism, while different mechanisms can be used to realize the same behavior.

\medskip

The mechanism is the proximate explanation; i.e. ``how?'', while the behavior is the ultimate explanation; i.e. ``why?''.
\end{center}
\end{flashcard}

\begin{flashcard}[Question]{In what four cases does cognition guide action?}
\begin{itemize}
\item When there is missing information,
\item When there is uncertain information,
\item When there is changing information,
\item When there is late information.
\end{itemize}
\end{flashcard}

\begin{flashcard}[Question]{What is the difference between Marr's and Kelso's hierarchies of abstraction?}
\begin{center}
In \textbf{Marr's} hierarchy of abstraction there is a \textit{loose} coupling between the computational theory, the representation and algorithms, and the hardware/software implementation, where the computational theory is the most important. The problem should be modelled first at the level of the computational theory without reference to lower levels. Marr compares his view to first understanding aerodynamics to understand flight, instead of studying bird feathers.

\medskip

In \textbf{Kelso's} hierarchy of abstraction there is a \textit{strong} coupling between the goals and tasks of the system, the system behavior, and the realized system, and all components are equally important. For Kelso, embodiment is integral to the model, and instantiation of the system has a direct role to play in the model itself. If you take away the context, you take away the basis for the model.
\end{center}
\end{flashcard}

\begin{flashcard}[Question]{What are four common ways of interacting with users besides the keyboard and mouse?}
\begin{itemize}
\item Natural language
\item Multi-touch
\item Gestures
\item Electroencephalography (EEG)
\end{itemize}
\end{flashcard}

\begin{flashcard}[Question]{What are potential benefits to intelligent user interfaces (IUI)?}
\begin{itemize}
\item More efficient interaction -- enabling more rapid task completion with less work.
\item More effective interaction -- doing the right thing at the right time, tailoring the content and the form of the interaction to the context of the user and task.
\item More natural interaction -- supporting human-like spoken, written, and gestural interaction.
\end{itemize}
\end{flashcard}
